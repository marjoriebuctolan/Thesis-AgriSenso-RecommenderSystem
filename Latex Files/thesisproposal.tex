
\documentclass{report}
%%% Import Packages %%%
\usepackage [a4paper, portrait, margin=1in]{geometry}
\usepackage{adjustbox}
\usepackage[utf8]{inputenc}
\usepackage{amsmath}
\usepackage{amsfonts}
\usepackage{amssymb}
\usepackage{fancyhdr}
\usepackage{graphicx}
\usepackage{booktabs}
\usepackage{array}
\usepackage[T1]{fontenc}
\usepackage{graphicx}
\usepackage{listings}
\usepackage{booktabs}
\usepackage{hyperref}
\usepackage{wrapfig}
\usepackage{lscape}
\usepackage{rotating}
\usepackage{epstopdf}
%%%

%%% Header and Footer Text %%%
\pagestyle{fancy}
\fancyhf{}
\fancyhead[LE,RO]{Agrisenso}
\fancyhead[RE,LO]{Documentation}
\fancyfoot[CE,CO]{\leftmark}
\fancyfoot[LE,RO]{\thepage}

\renewcommand{\headrulewidth}{2pt}
\renewcommand{\footrulewidth}{1pt}

%%% 

%%% Typesetting %%%
\linespread{1.20}
%%%

\begin{document}
\title{\textbf{Thesis Proposal Documentation} \\ 
}
\author{
\normalsize
Buctolan, Marjorie \\
Teves, Shannon Cates \\
}


\pagenumbering {gobble}
\maketitle
\tableofcontents
\newpage
\pagenumbering {arabic}

\begin{abstract}
Recommendation System has been popular nowadays it has been used for well-known
e-commerce sites and software applications like Amazon, Lazada, Netflix, and Youtube. It is
used to predict the rating or preference that a user would give to an item. The focus of this
study is to implement a recommendation system in AgriSenso, an agricultural web-based
application that help and connect buyers/consumers to the sellers/farmers and vice versa. It is
also a guide for managing a farm and for growing farm products. The researchers identify
different approaches of algorithm techniques used by recommender system such as
Collaborative Filtering we used Item-Based Collaborative Filtering Recommendation Algorithms
to compare the items that the two target users had rated and compute the similarities of the
items ranking. A pathfinding algorithm, A* Search Algorithm was also used for searching all
possible nearest seller/farmer from the buyer.
\end{abstract}

\chapter{Research Description}
\label{chpt: Research Description}
Recommender systems are software applications that make product/service recommendations
with the aim of optimizing some user-oriented objective in light of inherent uncertainty over
users and contents (Shengbo Guo). Recommendation is widely used technique to guide user to
choose the right product online as it becomes the essential feature for the successful
E-commerce application. The modern E-commerce system widely used good recommendation
system which will not only guide user to choose correct product at correct time but also it helps
attracting the customer to do business online effectively. Recommender systems increase
interaction to provide a richer experience of the users. To create recommender systems, a
reliable algorithm is needed to implement.
\section{Background of the Study}
The AgriSenso system is a web-based application ​ that helps seller to find potential buyer of
their products regularly. It is a buy-and-sell platform that connects both seller and buyer
through the system or simply via chat or phone call. Also, a guide for managing a farm and for
growing farm products.
Moreover, a recommender system will be implemented in AgriSenso system. Collaborative
filtering (CF) at a recommendation is based on a model of prior user behavior. Item-Based
collaborative filtering algorithm is an approach of CF that will recommend the best seller, best
buyer, and best product. This approach looks into the set of items and then selects the most
similar items. A* algorithm is used to search paths where it moves from the starting point to
the goal to find the length of the shortest path. This method will recommend the nearest
location of the seller/farmer from the buyer. 

\section{Statement of the Problem}
As a result, in the conducted survey to know the current state of the farmers in Iligan City
specifically in rural areas and also to the potential buyers of the area, researchers found out
that the most common problems that they have faced are as follows:\\ \\
Seller/Farmer:\\ \\
a. Farm product underpricing offered by buyers.\\
b. Difficulties on finding potential buyers of their farm products.\\
c. Difficulties on selling farm products.\\
d. Doesn't have a guide to manage their farm products and they don't know what are the
best farm products they will raise at a specific time.\\ \\
Buyer:\\ \\
a. Difficulties on finding suppliers/sellers to provide them farm products that they want.\\
b. Difficulties on finding the nearest suppliers/sellers.\\
c. Difficulties on finding other products that may be useful to the buyer.\\

\section{Research Objectives}
\subsection{General Objectives}
To implement different approaches of algorithm techniques used in AgriSenso’s
recommendation system. A* Search algorithm technique finds the nearest location of the buyer
to the seller. Item-Based collaborative filtering technique recommend product to the user
through the similarities of the purchased item of the other user. Individual user’s rank to each
item will be used as a data to calculate the similarities of each item that each user purchased.

\subsection{Specific Objectives}
The following are the specific objectives of the research: 
\begin{enumerate} 
	\item To conduct a survey to local farmers especially in rural areas to gather informations.
	\item To build a web-based system ‘AgriSenso’ the features to be implemented will only focus
on those necessary features affected by the recommender system.
	\item To implement the best seller recommender system, where buyers can be recommended
sellers based on the biggest number of farm products bought from its different buyer
wherein CF algorithm in Item-based approach can be applied.
	\item To implement the best buyer recommender system, where sellers can be recommended
buyers that has the highest rating given by the sellers wherein CF algorithm in
Item-based approach can be applied.
	\item To implement the best similar product recommender system, where buyers can be
recommended products that has similar preferences of items based on the rating
patterns of items given by the buyers wherein CF algorithm in Item-based approach can
be applied.
	\item To implement the nearest location recommender system, where seller’s nearest
location can be recommended by buyers and vice versa wherein A* search algorithm
can be applied.
\end{enumerate}

\section{Scope and Limitations of the Research}
Recommender system(RS) will be implemented in this research. These are recommendations
for the best product, best seller, the best buyer, and the nearest location of the seller from a
buyer. The recommender system for the best seller will be based by the number of farm
products bought by the buyer. The researchers will get data from its system database. Therecommender system for the best buyer and best product will be based by the rating given by
the seller. To control the data gathering process, the researchers will create an account for each
test user(seller) and will provide the list of buyers and list of products that they will going to
rate. An interview will be conducted to each test user so that they will have an idea as to what
specific data to provide. The test user will rate a buyer from one star to five stars, one star will
be the lowest and five stars will be the highest. The recommender system for the nearest
location will only be provided locally in the Philippines. Also, test user will be needed in this RS.
Test user should provide his/her exact location.

\section{Significance of the Research}
AgriSenso system serves as a bridge for both sellers/farmers and buyers. Basically, it fills the
gap between farmers and agriculture experts. This research generates recommender system
that can provide user a scalable way to help discover what product they may like and who is the
good or maybe the best seller or best buyer to get accounted with. Therefore, this research can
help users make good and faster decisions.


\chapter{Review of Related Literature}
\label{chpt: Review of Related Literature}
\section{Item-Based Collaborative Filtering}
Due to the tremendous growth in the amount of available information and the number of
visitors on Websites, new recommender system technologies are needed that can quickly
produce high quality recommendations, even for very large-scale problems. To address these
issues Sarwar, Karypis, Konstan, and Riedl explored item-based collaborative filtering
techniques. The research paper focused on the analyzation of different item-based
recommendation generation algorithms and different techniques for computing item-item
similarities. The researchers evaluate a new algorithm for CF-based recommender systems and
compare the results to the basic k-nearest neighbor approach. According to the researchers
their experiment suggest that item-based algorithms provide better performance and quality
than user-based algorithms for item-based techniques hold the promise of allowing CF-based
algorithms to scale to large data sets.
Another study, Item-Based Top-N Recommendation Algorithms of Mukund Deshpande
and George Karypis. The emergence of e-commerce sites and rapid growth of the World Wide
Web has led to the development of recommender systems. User-based collaborative filtering is
said to be the most successful technology for building recommender systems and is used in
many commercial recommender systems. Unfortunately, the computational complexity of
these methods grows linearly with the number of customers in commercial applications whichtypically runs several millions. Their article presented one such class of model-based top-N
recommendation
algorithms
that
use
item-to-item
similarities
to
compute
the
recommendations. The result of their experimental evaluation on eight real datasets shows that
“item-based algorithms are up to two orders of magnitude faster than the traditional
user-neighborhood based recommender systems”.

\section{A-Star(A*) Search Algorithm}
In the study of Xiao Cui and Hao Shi, A*-based Pathfinding in Modern Computer Games
explores the relationship between various A*-based algorithms and reviews a number of
popular A*-based algorithms and techniques according to the optimization of A*. They mention
some ways to improve the performance of A* search include optimizing the underlying search
space, reducing the memory usage, improving heuristic functions and introducing new data
structures. The researchers also define another way to make some contribution to the game AI
community to the real computer games.
The research paper of Hart, Nilsson, and Raphael they described how heuristic
information from the problem domain can be incorporated into a formal mathematical theory
of graph searching. And they demonstrated an optimality property of a class of search
strategies. The study concerned with the subgraph G from some single specified start node s.
An algorithms that search G to find an optimal path from s to a preferred goal node of s.

\section{Recommender Systems}
According to Beel, Gipp, Langer, and Breitinger more than half of the recommendation
approaches applied content-based filtering (55%). Collaborative filtering was applied by 18% of
the reviewed approaches, and Graph-based recommendations by 16%. The researchers
revealed some shortcomings of the current research such as it remains unclear which
recommendation concepts and approaches are the most promising. Different results on the
performance of content-based and collaborative filtering. They also discussed three potential
reasons for the ambiguity of the results. First is the several evaluations had limitations based on
strongly pruned datasets. Second, some authors provided little information about their
algorithms, resulting for the difficulties of re-implementing the approaches. And lastly, minor
variations in datasets, algorithms, or user populations inevitably lead to strong variations in the
performance of the approaches.


\chapter{Research Methodology}
\label{chpt: Research Methodology}
This chapter discusses the relevant theories and concepts used in this research.
\section{Survey}

\newpage
\section{System Architecture}

\newpage
\section{Item-Based Collaborative Filtering Algorithm}
\subsection{Cosine-Based Similarity}
The similarity between the two items are measured by computing the cosine of the angle
between the two vectors (where two items are two vectors in the m dimensional user-space).
\subsection{Correlation-Based Similarity}
The similarity between two items ​ i and​ j is measured by computing the Pearson-r correlation.
\subsection{Adjusted Cosine Similarity}
Offsets drawback by subtracting the corresponding user average from each co-rated pair.

\newpage
\section{A-Star(A*) Search Algorithm}
\end{document}
\grid